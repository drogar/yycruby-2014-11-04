%!TEX root = /Users/gilesb/programming/presentations/yycruby-2014-11-04/jruby-presentation.tex
\usepackage{etex}
\usepackage[english]{babel}
% or whatever

\usepackage[all]{xy}
\usepackage[latin1]{inputenc}
% or whatever

\usepackage{palatino,courier}
\usepackage{amsfonts,amssymb}
\usepackage[mathscr]{euscript}
\usepackage{stmaryrd}
\usepackage{eulervm}

\usepackage{proof}
\usepackage{amsmath}
\usepackage{xspace}

\usepackage{graphicx}
\usepackage{tikz}
\usepackage{listings}
\usepackage{color}

\lstset{language=Ruby}
\usetikzlibrary{calc,arrows,positioning,shapes.geometric,fit,decorations.markings}

\pgfdeclarelayer{edgelayer}
\pgfdeclarelayer{nodelayer}
\pgfsetlayers{edgelayer,nodelayer,main}

\tikzset{ node distance=.1mm, inner sep=0.5mm}

\tikzstyle{delta}=[isosceles triangle, isosceles triangle apex angle=70, draw, shape border rotate=90, minimum size=2mm]
\tikzstyle{eta}=[circle, draw, minimum size=2mm]
\tikzstyle{epsilon}=[circle, draw, fill=black!75, minimum size=2mm]
\tikzstyle{nabla}=[isosceles triangle, isosceles triangle apex angle=70, draw, shape border rotate=270, minimum size=2mm]
\tikzstyle{map}=[rectangle,draw]
\tikzstyle{nothing}=[rectangle,draw=white]


\input{macrospres}

\title{Why JRuby, Why?!}

\author[Brett~Giles]{\protect{Brett~Giles}}

% - Use the \inst{?} command only if the authors have different
%   affiliation.

\institute[YYCRuby] % (optional, but mostly needed)
{  YYCRuby Meetup}
% - Use the \inst command only if there are several affiliations.
% - Keep it simple, no one is interested in your street address.

\date
{2014-11}

\subject{JRuby}
% This is only inserted into the PDF information catalog. Can be left
% out.

% Delete this, if you do not want the table of contents to pop up at
% the beginning of each subsection:
%\AtBeginSubsection[]
% {
%  \begin{frame}<beamer>{Outline}
%    \tableofcontents[currentsection,currentsubsection]
%  \end{frame}
% }


% If you wish to uncover everything in a step-wise fashion, uncomment
% the following command:

%\beamerdefaultoverlayspecification{<+->}


\begin{document}

\begin{frame}
  \titlepage
\end{frame}

\section{What is JRuby?}
\begin{frame}\frametitle{Ruby on Java!}
\begin{itemize}
  \item It \emph{is} just ruby. Write it. Run it.
  \item .... but... you \emph{can} call Java.
  \item .... and... you \emph{can't} call C.
\end{itemize}
\end{frame}
\begin{frame}\frametitle{Why call Java?}
  \begin{itemize}
    \item Threads!
    \item WARs!
    \item Big conservative companies!
    \item Swing!
  \end{itemize}
\end{frame}
\begin{frame}\frametitle{Why \emph{not} call Java?}
  \begin{itemize}
    \item Context switching (yours - not the computer's).
    \item JVM startup is a pain (unTDD-like). (Spork, NG-Server are answers).
    \item Tends to lag behind. Current production release is 1.9 compatible. 2.2 coming soonish...
    \item You really really \emph{really} need the C interface. (Try harder not to :)
  \end{itemize}
\end{frame}
\begin{frame}\frametitle{Threads}
  The dreaded Global Interpreter Lock!
  \begin{center}
  \includegraphics[scale=.5]{diagrams/ruby-gil.png}
  \end{center}
  But... Rubinius does not have a GIL either....


{\tiny Diagram from https://www.igvita.com/2008/11/13/concurrency-is-a-myth-in-ruby/}
\end{frame}
\begin{frame}[fragile]\frametitle{Thread code - Ruby}
\begin{lstlisting}
num_iterations.times do |iter|
  threads = []
  t_0 = Time.now
  (1..num_threads).each do |i|
    threads << Thread.new(i) do
      count = 0
      1_000_000.times { count += 1 }
    end
  end
  threads.each(&:join)
end
\end{lstlisting}
\end{frame}
\begin{frame}[fragile]\frametitle{Thread code - JRuby - Pg 1}
\begin{lstlisting}
class CountMillion
  include Callable
  def call
    count = 0
    1_000_000.times { count += 1 }
  end
end
executor = ThreadPoolExecutor.new(
   num_threads, # core_pool_treads
   num_threads, # max_pool_threads
   60, # keep_alive_time
   TimeUnit::SECONDS,
   LinkedBlockingQueue.new)
\end{lstlisting}
\end{frame}

\begin{frame}[fragile]\frametitle{Thread code - JRuby - Pg 2}
\begin{lstlisting}
num_iterations.times do |i|
  tasks = []

  t_0 = Time.now
  num_threads.times do
    task = FutureTask.new(CountMillion.new)
    executor.execute(task)
    tasks << task
  end

  # Wait for all threads to complete
  tasks.each(&:get)
end
\end{lstlisting}
\end{frame}
\begin{frame}\frametitle{Thread Demo}
  \begin{center}
  {\Huge DEMO}
  \end{center}
\end{frame}
\begin{frame}\frametitle{WARs and BCCs}
  ...or ``How to use Rails in a 6000+ person O\&G company''.
  \begin{itemize}
    \item JRuby on Rails works. Started using it at 2.x, did some 3, other have done 4.
    \item A WAR is a Web ARchive and is how Java Web apps are deployed on a server, e.g. Tomcat.
    \item \url{github.com/jruby/warbler} will package your app into a warfile.
    \item Can use Trinidad as lightweight option --- one java process, many threads.
  \end{itemize}
  When you are in a large conservative company, all of the above is a GOOD THING!
\end{frame}
\begin{frame}\frametitle{JRuby on Rails}
  Need to do a few things....
  \begin{itemize}
    \item Use \texttt{therubyrhino} instead of \texttt{therubyracer}
    \item Use \texttt{activerecord-jdbc-adapter} plus db of choice.
    \item Check the jruby wiki for alternatives if using gems with C-Extensions.
    \item Can use on Heroku!
  \end{itemize}
  Of course, must have the capability to get deploy to happen by just supplying a WAR. (Bigger
  companies may insist on building by themselves.)
\end{frame}
\begin{frame}\frametitle{Niceties of JRoR}
  We started with JRoR as it was the only choice --- The nice things about this:
  \begin{itemize}
    \item Fast. Speedy. Zoom! (After startup)
    \item Threads - Had to be a bit careful, but Rails 4 is thread-safe.
    \item Memory usage is good, scales well.
  \end{itemize}
\end{frame}
\begin{frame}\frametitle{Quotes and resources}
  On a web application that took advantage of the shared memory capabilities to provide responsive
  processing:
  \begin{quote}
  ``Teams have tried and failed to write an application that can handle the demands of [Cooperative
    Speculative planning], this [JRuby] application exceeds all of our expectations''
  \end{quote}
  Resources:
  \begin{itemize}
    \item PragProg: ``Using JRuby'' and ``Deploying with JRuby''.
    \item \href{https://github.com/jruby/jruby/wiki}{\textcolor{blue}{The wiki}} and
      \href{http://jruby.org}{\textcolor{blue}{the JRuby website}}.
    \item Twitter: \href{https://twitter.com/headius}{\textcolor{blue}{@headius}},
      \href{https://twitter.com/tom\_enebo}{\textcolor{blue}{@tom\_enebo}},
      \href{https://twitter.com/jruby}{\textcolor{blue}{@jruby}}.
  \end{itemize}
\end{frame}
\begin{frame}\frametitle{and finally, swing, swing, swing}
  Why? (no, really, why?)
  \begin{itemize}
    \item I had to convert a Haskell GTK based GUI to something people could build on their own.
    \item It had to be cross platform.
    \item I liked Ruby.
  \end{itemize}
  Options for swing:
  \begin{itemize}
    \item No framework - just use the swing classes and methods.
    \item Rubeus - Adds block to the swing class constructors.
    \item MonkeyBars - Encourages MVC design, has generators!
  \end{itemize}
\end{frame}
\begin{frame}\frametitle{actually using it...}
  \begin{itemize}
    \item I use rake, rspec, cucumber, simplecov, flog, flay, rubocop and guard as part of the dev process.
    \item I spent WAAAY too much time getting cucumber to work with swing.
    \item \texttt{jruby --ng-server} can be your friend.
    \end{itemize}
\end{frame}
\begin{frame}\frametitle{Swing Demo}
  \begin{center}
  {\Huge DEMO}
  \end{center}
\end{frame}

\end{document}



%%% Local Variables:
%%% mode: latex
%%% TeX-master: "withnotes"
%%% End:
